\chapter{Resultados}
\chaptermark{Resultados}

\section{Aplicación de referencia}

Como ya se ha descrito en anteriores secciones, los sistemas guest desarrollados para evaluar el rendimiento de los hipervisores Xen y Jailhouse, tienen como obetivo medir la latencia introducida en la atención a interrupciones debida a la utilizaión de los mismos. \\
Con objeto de poder comparar los resultados obtenidos, es necesario establecer unos datos de partida, es decir, caracterizar el comportamiento del sistema cuando no hay ningún hipervisor. Para ello, se ha desarrollado una aplicación que se ejecuta de forma nativa en la tarjeta electrónica sin ningún sistema operativo y sin hipervisor, lo que se conoce tradicionalmente por baremetal.

El código fuente de esta aplicación es exactamente el mismo que el del DomU de la sección \ref{}

De esta forma, con la diferencia en la cuenta del temporizador entre el instante en el que se produjo la interrupción y el intante en el que fue recibida y reconocida, marca la latencia

\section{Aplicación baremetal en celda Jailhouse}

\section{Aplicación baremetal en DomU de Xen}

\pgfplotstableread[row sep=\\,col sep=&]{
    interval & carT & carD & carR \\
    0--2     & 1.2  & 0.1  & 0.2  \\
    2--5     & 12.8 & 3.8  & 4.9  \\
    5--10    & 15.5 & 10.4 & 13.4 \\
    10--20   & 14.0 & 17.3 & 22.2 \\
    20--50   & 7.9  & 21.1 & 27.0 \\
    50+      & 3.0  & 22.3 & 28.6 \\
    }\mydata

\begin{tikzpicture}
    \begin{axis}[
            ybar,
            bar width=.5cm,
            width=\textwidth,
            height=.5\textwidth,
            legend style={at={(0.5,1)},
                anchor=north,legend columns=-1},
            symbolic x coords={0--2,2--5,5--10,10--20,20--50,50+},
            xtick=data,
            nodes near coords,
            nodes near coords align={vertical},
            ymin=0,ymax=35,
            ylabel={\%},
        ]
        \addplot table[x=interval,y=carT]{\mydata};
        \addplot table[x=interval,y=carD]{\mydata};
        \addplot table[x=interval,y=carR]{\mydata};
        \legend{Trips, Distance, Energy}
    \end{axis}
\end{tikzpicture}
