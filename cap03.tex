\chapter{Entorno de desarrollo}
\chaptermark{Entorno de desarrollo}

A la hora de seleccionar la plataforma para desarrollar las pruebas sobre los hipervisores Xen y Jailhouse se han barajado diferentes alternativas. Los criterios que se han tenido en cuenta para la selección han sido principalmente dos:
\begin{itemize}
  \item Plataformas soportadas: cada uno de los dos hipervisores mantiene una lista de plataformas en las que se ha probado alguna vez. Xen tiene un amplio recorrido en el mundo de los hipervisores y la lista de tarjetas electrónicas en las que se ha probado es muy grande. En el caso de Jailhouse, debido a que su trayectoria es más corta, esa lista de tarjetas soportadas se reduce enórmemente \cite{jailhouse_github}. La tarjeta seleccionada debía estar en las dos listas.
  \item Herramientas para generar máquinas virtuales: con el objeto de evaluar un sistema virtualizado con ambos hipervisores, se va a necesitar generar tanto un dominio o celda Linux con su kernel, device-tree y sistema de ficheros y un sistema baremetal. La selección de la tarjeta electrónica, o más bien la arquitectura del microprocesaro que incluya, marca también en gran medida las herramientas (compilador, BSP, etc.) para generar el software necesario. Cabe destacar que se le ha dado importancia al hecho de poder disponer de todo el código fuente que se vaya a utilizar para poder controlarlo, entenderlo y adaptarlo si fuera necesario.
\end{itemize}

Una de las arquitecturas más populares en los últimos tiempos en los sitemas embebidos es la arquitectura ARM. Es ubicua y en los ultimos años ha irrumpido con mucha fuerza. La famosa tarjeta de la frambuesa desde su inicio ha incluído un procesador ARM, lo que le ha dado más popularidad si cabe, y también Xilinx desde la familia Zynq\textregistered-7000 en su PS.\\
Debido a la familiaridad con las plataformas de Xilinx y sus herramientas y al creciente interés en la virtualización sobre la arquitectura ARMv8, se decidió optar por una tarjeta que incluyera un chip de la familia Zynq\textregistered UltraScale+\texttrademark. Además de esto, resulta de especial interés que además de incluir un procesador ARMv8 en la zona de PS, incluye una parte de PL o FPGA en la que se pueden desplegar diferentes periféricos con el objetivo de establecer un entono de pruebas y medidas flexible.

En el momento de iniciar el trabajo fin de máster no se disponía de la tarjeta Xilinx ZCU102, que aparece en la lista de plataformas soportadas en Xen y Jailhouse. En cambio, la platarforma que se podía utilizar era la UltraZed\texttrademark y debido a que incluye un chip de la familia Zynq\textregistered UltraScale+\texttrademark MPSoC al igual que la ZCU102, se estimó que las modificaciones necesarias para poder ejecutar ambos hipervisores no serían demasiado costosas, por lo que el desarrollo se ha efectuado sobre esa plataforma.

\section{Plataforma electrónica}
\subsection{Ultrazed}
La tarjeta electrónica que se ha utilizado se compone de dos partes: UltraZed-EG\texttrademark SOM y UltraZed\texttrademark IO Carrier Card con las siguientes características técnicas:
\begin{itemize}
  \item Procesador UltraScale+\texttrademark MPSoC XZU3EG-1SFVA625E que inluye:
  \begin{itemize}
    \item 4 núcleos ARM Cortex-A53 (ARMv8) hasta 1.2 GHz
    \item 2 núcleos ARM Cortex-R5 hasta 500 MHz.
    \item Procesador gráfico Mali™-400 MP2 hasta 600 MHz
    \item 154K Celdas lógicas
    \item 141K Flip-Flops
    \item 7.6 Mb \acrshort{BRAM}
    \item 360 bloques \acrshort{DSP}
    \end{itemize}

    \begin{figure*}[h]
    	\centering
    	\includegraphics[width=0.75\textwidth]{recursos/mpsoc_arch.png}
    	\caption{Bloques de MPSoC EG}
    	\label{fig:mpsoc_arch}
    \end{figure*}

  \item 2 GB \acrshort{DDR}4 \acrshort{SDRAM}
  \item 64 MB memoria flash \acrshort{QSPI}
  \item 8 GB memoria flash \acrshort{eMMC}
  \item 1 Gigabit Ethernet
  \item 1 conector para tarjetas microSD
  \item 12 conectore \acrshort{PMOD} conectados a la parte de \acrshort{PL}
  \item 1 \acrshort{DIP} switch de 8 posiciones conectado a \acrshort{PL}
  \item 8 LEDs conectados a \acrshort{PL}
  \item 1 conector \acrshort{JTAG}
  \item 2 \acrshort{USB}-\acrshort{UART}
\end{itemize}

\begin{figure*}[h]
	\centering
	\includegraphics[width=0.65\textwidth]{recursos/ultrazed-eg-carrier.png}
	\caption{UltraZed\texttrademark IO Carrier Card}
	\label{fig:ultrazed-eg-carrier}
\end{figure*}

\subsection{Otros}

\begin{itemize}
  \item Analizador lógico de 8 canales: con el objeto de realizar algunas medidas empíricas de tiempos de reacción de los hipervisores se ha utilizado como ayuda el siguiente analizador:\\
  \url{https://eur.saleae.com/products/saleae-logic-8}
  \\Algunas de las medidas se han realizado con temporizadores como se verá posteriormente. Aún así el analizador ha servido de apoyo a la hora de comprobar que los resultados obtenidos de los temporizadores eran correctos.
  \item Cable JTAG-HS2: a fin de cargar de forma dinámica y depurar los programas que se han hecho se ha empleado el siguiente debugger aptos para productos de Xilinx.\\

  \url{https://www.digikey.es/es/product-highlight/d/digilent/jtag-hs2-programming-cable}
\end{itemize}

\section{Herramientas Software}

A continuación se listan las herramientas software que se han empleado en la elaboración de los tests en los hipervisores Xen y Jailhouse. Para la plataforma electrónica seleccionada ha sido suficiente la licencia gratuita (WebPACK) que proporciona Xilinx.

\begin{itemize}
  \item Vivado Design Suite 2018.3: este paquete de software de Xilinx es el que permite generar los que se denomina la \textit{Hardware Platform} y exportala a fin de poder crear proyectos de software sobre ella como se verá más adelante. Esta \textit{Hardware Platform} contiene tanto la parte de configuración de la parte de PS del Zynq\textregistered UltraScale+\texttrademark MPSoC como el bitstream que se programa en la zona PL.\\
  \url{https://www.xilinx.com/products/design-tools/vivado.html}
  \item Xilinx Software Development Kit (XSDK) 2018.3: la herramienta \acrshort{XSDK} es un entorno inetrado de desarrollo basado en eclipse que permite desarrollar aplicaciones para ejecutarse sobre la plataforma generada por Vivado. Mediante un sencillo interfaz gráfico se pueden crear aplicaciones tanto baremetal, como FreeRTOS y Linux. Soporta todas las familias de microprocesadores existentes en las plataformas de Xilinx: MicroBlaze, ARM Cortex-R5, ARM Cortex-A9 y ARM Cortex-A53. Cabe destacar que también inluye un sistema de depurado de software y herramientas de análisis de rendimiento.\\
  \url{https://www.xilinx.com/products/design-tools/embedded-software/sdk.html}
  \item Petalinux Tools 2018.3: en este paquete se proporcionan todas las herramientas y componentes software para construir un sistema embebido basado en Linux en las plataformas de Xilinx.\\
  Desde las últimas versiones su sistema de compilación y generación de software está basado en el proyecto yocto (\url{https://www.yoctoproject.org/}). Xilinx añade una capa software sobre yocto para manejar las particularizaciones de sus productos.\\
  \url{https://www.xilinx.com/products/design-tools/embedded-software/petalinux-sdk.html}
\end{itemize}
