\section{Introduccion}
\label{ch:introduccion:primera}
La virtualización es un conjunto de técnicas hardware y/o software que permiten la ejecución
concurrente de instancias aisladas de distintos sistemas operativos en una misma plataforma
electrónica. A estos sistemas operativos se les denomina guest.
Los hipervisores fueron originalmente introducidos en el mundo IT para solucionar los problemas de
balanceo de carga y utilización de recursos en data centers. En sus inicios, los hipervisores necesitaban cambios en
los sistemas operativos guest para compensar la falta de soporte hardware para el aislamiento
entre sistemas. A medida que las arquitecturas de micorprocesadores han ido avanzando y añadiendo
soporte hardware para la virtualización, los hipervisores se han vuelto cada vez más habituales en los
sistemas embebidos.

La inclusión de soporte hardware para la virtualización dentro de las nuevos microprocesadores ha sido
el habilitador necesario para que la virtualización de el salto del mundo IT a los sistemas embebidos. Las
arquitecturas más importantes de microprocesadores han evolucionado hacia la inclusión de ése soporte hardware.
Como ejemplos notables están Intel VT-x, las extensiones de virtualización de la arquitectura ARM y MIPS VZ
extensions.
Este soporte hardware, lo que aporta es un modo de ejecuación distinto, con mayores privilegios que el tradicional modo
supervisor y IO MMUs para aislar periféricos entre sistemas operativos guest. La versión de intel de IO MMU recibe el
nombre de VT-d y en la mayoría de los sistemas ARM, existe un "System MMU". En los data center, el IO MMU se denomina habitualmente "Single Root Virtualization" o SRV.
Los sistemas operativos guest, al estar aislados unos de otros, hace que parezca que
están ejecutándose en máquinas físicamente distintas. La parte de software encargada de
monitorizar y gestionar las diferentes máquinas virtuales y el acceso que tienen a los recursos del
sistema es lo que se denomina hipervisor.
Existen diferentes tipos de virtualización y por tanto de hipervisores. Hay dos grandes grupos de
hipervisores:
\begin{itemize}
\item Tipo 1 o baremetal: en este grupo están los hipervisores que se ejecutan directamente sobre
la plataforma electrónica sin ningún otro software de por medio, tipo sistema operativo u
otros controladores. Tienen acceso directo al hardware.
\item Tipo 2: estos hipervisores suelen instalarse sobre un sistema operativo (host) de base y
acceden a los recursos del sistema (memoria, procesadores, almacenamiento, red, etc.) a
través de él.
\end{itemize}

\begin{figure*}[!htb]
	\centering
	\includegraphics[width=0.5\textwidth]{recursos/Figure2}
	\caption{Violinplot fot BESA in scenario 4}
	\label{fig:violin_besa_escenario4}
\end{figure*}

\begin{figure*}[!htb]
    \centering
    \begin{subfigure}[b]{0.5\textwidth}
        \includegraphics[width=\textwidth]{recursos/Figure1a}
        \caption{Mean cumulative regret along trials}
        \label{fig:Bernoulli1_semilog}
    \end{subfigure}
    \begin{subfigure}[b]{0.5\textwidth}
        \includegraphics[width=\textwidth]{recursos/Figure1b}
        \caption{Multiple violinplot}
        \label{fig:Bernoulli1_boxplot}
    \end{subfigure}
    \caption{Comparative of the policies for scenario 1}
    \label{fig:Bernoulli1}
\end{figure*}

\begin{figure}[!ht]
	\centering
	\input{recursos/variacion_fitness}
	\caption{Variación de la función objetivo en intervalos de 1000 iteraciones en el caso 8 para diferentes soluciones iniciales}
	\label{fig:variacion_fitness}
\end{figure}

\section{Expresiones matemáticas}
A continuación, se muestran algunos ejemplos de expresiones matemáticas:
\begin{equation}
\mu^*\times 25000-\frac{1}{1000}\sum_{r=1}^{1000}\sum_{i=1}^{K}\sum_{j=1}^{25000}\mu_i\times X_{i,j}^r.
\end{equation}

\begin{equation}
\mu_{\widetilde{A}}(x)=\left\{ \begin{array}{cc}
\frac{x-a_{1}}{a_{2}-a_{1}} & if\; a_{1}\leq x\leq a_{2}\\
1 & if\; a_{2}\leq x\leq a_{3}\\
\frac{x-a_{4}}{a_{3}-a_{4}} & if\; a_{3}\leq x\leq a_{4}\\
0 & otherwise
\end{array}\right. .
\end{equation}

\begin{align}
\begin{split}
\widetilde{DD}(A_{1},A_{4}) & =\widetilde{DD}(A_{1},A_{4}|P_{1})\oplus \widetilde{DD}(A_{1},A_{4}|P_{2}) \\
& =[\widetilde{dd}(A_{1},A_{2})\otimes \widetilde{dd}(A_{2},A_{4})]\oplus \lbrack \widetilde{dd}(A_{1},A_{3})\otimes \widetilde{dd}(A_{3},A_{4})].
\end{split}
\end{align}

\begin{itemize}
\item Si $\max \{(a_{4}-a_{1}),(b_{4}-b_{1})\}\neq 0$, entonces
\begin{align*}
S(\widetilde{A},\widetilde{B})=& 1 - ( 1 - \alpha - \beta) \times \left( 1 - \frac{\int_{0}^{1} \mu_{\widetilde{A}\cap \widetilde{B}}(x)dx} {\int_{0}^{1} \mu_{\widetilde{A}\cup \widetilde{B}}(x)dx}\right) \\
& -\alpha\frac{\sum \mid a_{i} - b_{i} \mid }{4}- \beta \frac{d[(X_{\widetilde{A} },Y_{\widetilde{A}}),(X_{\widetilde{B}} ,Y_{\widetilde{B}})]}{M},
\end{align*}

\item En caso contrario,%
\begin{align*}
S(\widetilde{A} ,\widetilde{B})=& 1- \left( \frac{1-\alpha-\beta}{2} + \alpha \right) \times
\frac{\sum \mid a_i - b_i \mid}{4} - \\
& - \left( \frac{1 - \alpha - \beta}{2} + \beta \right) \times \frac{d[(X_{\widetilde{A}},Y_{\widetilde{A}}), (X_{\widetilde{B}},Y_{\widetilde{B}})]}{M},
\end{align*}
\end{itemize}
donde $\alpha +\beta <1$, $\mu _{\widetilde{\chi }}$ es la función de pertenencia de $\widetilde{\chi}$,
\begin{equation}
M=\underset{[0,1]\times[0,\frac{1}{2}]}{\max}\{d((x,y),(x^{\prime },y^{\prime }))\}\text{,}
\end{equation}%
\begin{equation*}
\mu _{\widetilde{A}\cap \widetilde{B}}(x)=\underset{0\leq x\leq 1}{\min}%
\{\mu _{\widetilde{A}}(x),\mu _{\widetilde{B}}(x)\} ,
\;\;\; \mu _{\widetilde{A}\cup \widetilde{B}}(x)=\underset{0\leq x\leq 1}{\max}%
\{\mu _{\widetilde{A}}(x),\mu _{\widetilde{B}}(x)\}.
\end{equation*}%

\section{Algoritmos}

El Algoritmo \ref{getDelay} ilustra la forma que debe adoptarse. Se usa el paquete \textit{algorithmic}. Los comandos pueden consultarse en \url{https://en.wikibooks.org/wiki/LaTeX/Algorithms} o en la documentación oficial.

\begin{algorithm}[h]
	\begin{algorithmic}
	\REQUIRE $t_0$ = instante en el que se genera el retardo

	\IF{$(update\_architecture==1)$}
		\IF{$(delay\_scenario==1)$}
			\STATE{delay$=C$}
		\ELSE
			\IF{$(reward\_scenario==1)$}
				\STATE{delay $\leftarrow [0,300]$-trunc\_Exp($\lambda=1/80$)}
			\ELSE
				\STATE{delay $\leftarrow [0,480]$-trunc\_Exp($\lambda=1/150$)}
			\ENDIF
		\ENDIF
	\ELSE
		\STATE{delay = difference(24:00, $t_0$)}
	\ENDIF

	\RETURN delay
	\end{algorithmic}
	\caption{$getDelay(t_0)$}
	\label{getDelay}
\end{algorithm}

\section{Tablas}
La tabla \ref{table:results45} muestra un ejemplo de cómo poner notas a pie de tabla (necesario el paquete threeparttable). En la tabla \ref{table:risk} se usa el entorno \textit{stripedtable} para marcar la líneas alternas con diferentes colores.

\begin{table}[htb]
\begin{threeparttable}
	\centering
	\caption{Mean cumulative regrets and standard deviations}
	\label{table:results45}
	\begin{small}
	\begin{tabular}{llllll}
		\toprule
			& \multicolumn{2}{c}{Truncated Poisson} & & \multicolumn{2}{c}{Truncated Exponential} \\
		\cmidrule{2-3}\cmidrule{5-6}
		 	& Mean & $\sigma$ & & Mean & $\sigma$\\ \midrule
		 UCB\tnote{1} & 2632.65 & 246.03 & & 1295.79 & 514.03 \\
		 DMED+ & 978.56 & 225.24 & & \textbf{645.70} & 493.8 \\
		 KL-UCB & 1817.4 & 236.57 & & 1219.98 & 510.69 \\
		 KL-UCB poisson & \textbf{314.99*} & 201.79 & & - & - \\
		 KL-UCB exp & - & - & & 786.30 & 498.16 \\
		 KL-UCB+ & 1190.64 & 225.82 & & 813.45 & 494.59 \\
		 BESA & 2015.73 & 3561.5 & & 755.87 & 2323.22 \\
		 PR-1 & 1314.9 & 234.25 & & 660.64 & 492.37 \\
		 PR-2 (TS) & \textbf{917.67} & 222.79 & & \textbf{630.38} & 487.01 \\
		 PR-3 & \textbf{736.6} & 210.96 & & \textbf{565.79*} & 480.99 \\
		 \bottomrule
	\end{tabular}
	\end{small}
	\begin{tablenotes}
		\item[1] Esto es una nota a pie de tabla
	\end{tablenotes}
\end{threeparttable}
\end{table}

\begin{stripedtable}[htb]
	\centering
	\caption{Risks to $A_5$ after the implementation of the selected safeguards}
	\label{table:risk}
	\begin{scriptsize}
	\begin{tabular}{cccc}
		\toprule
		\hiderowcolors
		Threat & Confidentiality & Integrity & Authenticity \\
		\midrule
		\showrowcolors
		$T_{1}^{1}$ & (16.9, 161.72, 936.2, 3681.5) & (32.70, 239.7, 1295.6, 5197.4) & (25.1, 198.6, 1576.7, 5777.1)\\
		$T_{1}^{2}$ & (0, 49.6, 458.1, 1791.2) & (0, 29.7, 289.7, 1397.1) & (0, 24.6, 352.6, 1552.9)\\
		$T_{2}^{2}$ & (0, 49.6, 458.1, 1791.2) & (0, 29.7, 289.7, 1397.1) & (76, 379.3, 2074.3, 5588.4)\\
		$T_{1}^{3}$ & (12.2, 110.5, 647.2, 2465.6) & (21.9, 147.3, 744.3, 2958.7) & (6.8, 58.5, 487.1, 1923.2)\\
		$T_{1}^{4}$ & (34.8, 245.5, 1176.8, 3793.2) & (62.7, 327.4, 1353.3, 4551.9) & (19.5, 129.9, 885.7, 2958.7)\\
		\bottomrule
		\hiderowcolors
	\end{tabular}
	\end{scriptsize}
\end{stripedtable}
